\documentclass[12pt]{article}
\usepackage{graphicx} %graphics envoirnment creation
\usepackage{tocbibind}
\pagenumbering{roman} %Setting the Page number
\usepackage{color}%to use color
\usepackage{hyperref} %Defining the subsections as links
\usepackage{fancyhdr}%to use header and footer in file
\usepackage{caption}
\usepackage{subcaption}
\usepackage{float}
\begin{document}



\begin{center}
{
{\huge{\textbf{Assingment Report}}}  %defining Front page
\textbf{TELECOMMUNICATION SOFTWARE LAB} \\
\textcolor{black}{\rule{\textwidth}{2pt}}
1st SEMESTER\\
2016\\
\bigskip  %for skipping lines
\bigskip
\bigskip
\bigskip
\bigskip
\bigskip
\bigskip
AKARSH AGRAWAL\\
2016JTM2092\\

\bigskip
\bigskip
\large{
Programming Assingment :7\\
Due Date:12.09.2016\\
}
}
\vspace{9mm} %setting vertical space
\includegraphics[width=4cm, height=4cm]{log.png} %entering logo
\end{center}




\newpage
\begin{center}
\renewcommand*\contentsname{Table of Contents} %Renaming Contents to Table of contents
 %Defining Table of contents
\tableofcontents
\end{center}

\newpage %For new page
\large{
\begin{center}
\listoffigures
\end{center}

\newpage
\begin{center}
\pagenumbering{arabic} %changing the page numbering style
\section{Introduction} %Defining sections 
\end{center}
\textbf{Python (programming language):}Python is a widely used high-level, general-purpose, interpreted, dynamic programming language. Its design philosophy emphasizes code readability, and its syntax allows programmers to express concepts in fewer lines of code than possible in languages such as C++ or Java. The language provides constructs intended to enable writing clear programs on both a small and large scale.\\
Python supports multiple programming paradigms, including object-oriented, imperative and functional programming or procedural styles. It features a dynamic type system and automatic memory management and has a large and comprehensive standard library.\\Python interpreters are available for many operating systems, allowing Python code to run on a wide variety of systems. Using third-party tools, such as Py2exe or Pyinstaller, Python code can be packaged into stand-alone executable programs for some of the most popular operating systems, so Python-based software can be distributed to, and used on, those environments with no need to install a Python interpreter.\\CPython, the reference implementation of Python, is free and open-source software and has a community-based development model, as do nearly all of its variant implementations. CPython is managed by the non-profit Python Software Foundation.





\newpage
\section{Problem statement1} %Defining sections 
Write a Python program that can take a big string (with spaces) as input from the command line and count number of times a word occurs in the string and also print the top 3 words in terms of their frequency of count.
Also print the next permutation of each word appearing in the string.

{

\subsection{Implementation problem1}
\begin{enumerate}
\item{Input the string using the rawinput}
\item{Split string in to words so that each could act as list }
\item{Count the word occurences in the string using loop and count prebuild functions}
\item{append the values as dictionary  }
\item{sort the dictionary and print the last three elements}
\item{Use pemutation inbuilt function}

\end{enumerate}
 \newpage
\subsection{Input}   %Defining subsections
my my name name is akarsh akarsh akarsh 


\subsection{Output}
akarsh: 3\\
my: 2\\
name: 2\\
ym\\
ym\\
naem\\
naem\\
si\\
akarhs\\
akarhs\\
akarhs\\

 
\begin{figure}[H]
{
 \centering
\includegraphics[scale=0.8]{output1.png}
\caption{sorted and pemutated words}
\vspace{2cm}

}
\end{figure}



\newpage
\section{Problem statement2} %Defining sections 

You are designing a Graphical user Interface (GUI) to depict the location of a mobile user in a square whose corner points are (1,1) (-1,1) (1,-1)(-1,-1). In real life, the user’s location would come from a database available with the MSC. For the moment, generate the user location using the random function generator function in Python to generate a number between [0,1). 

Using following code generate points inside this 2D shape.(import random)

 (X,Y)=(random.random()*2- 1, random.random()*2-1)

 Here, each point in above shape has an equal chance of being generated.
Finally calculate number of points that lie inside unit radius circle in terms of percentage.

\subsection{Implementation problem2}
\begin{enumerate}
\item{Generate the 100 users for example using random fucntion}
\item{check for the condition in circle equation whether point lie inside or not}
\item{if point lie inside increment teh counter and outpout the percentage}
\end{enumerate}
{



\subsection{Output}
\begin{figure}[H]
{
 \centering
\includegraphics[scale=0.55]{output2.png}
\caption{no of users calculation within unit radius}
\vspace{2cm}

}
\end{figure}



\newpage
\section{Problem statement3} %Defining sections 

You have to design an addressing code for a shipping company that works all around India. The address given by the customer is split into fields of name,city,district
Create a database with some default addresses.
The database should be editable(Add, delete, modify).
Also notify any discrepancy in data to the employee if the address is invalid or do not exist in the database.
First is machine readable like barcodes, in the form 1’s and 0’s as:\\
IIT Roorkee  = 001\\
Roorkee= 010\\
Uttarakhand = 100\\
Second is human readable, build by combination of first three letters of a place.\\
For example :\\
Prof. Ram Mishra\\
D - 15, North Enclave\\
IIT Roorkee, Roorkee\\
Uttarakhand\\
Hence the generated gives the collection center no. \\CCNO = 100010001\\
Hence human readable code\\ H-CC-NO = $ UTT-ROO-IIT-100010001$
\subsection{Implementation problem3}
\begin{enumerate}
\item{create a data base using doctionaries}
\item{map for the encoded part of the dictionaries values}
\item{ask for input from the user}
\item{is input matches with the user input route the map}
\item{output the encoded text both for user and machine}

\end{enumerate}
{

\subsection{Input}   %Defining subsections
kanpur\\
dhaulpur\\
UP\\

\subsection{Output}
\begin{figure}[H]
{
 \centering
\includegraphics[scale=0.55]{output3_2.png}
\caption{Enquiry output}
\vspace{2cm}

}
\end{figure}


\begin{center}
\newpage
\section{References and Citations}
\end{center}


\begin{thebibliography}{9}

 
\bibitem{grymoire}
python\\
 \texttt{https://www.python.org/}
 
 \bibitem{tutorial point}
python tutorials\\
\textit{https://www.codecademy.com/learn/python}

 
 
 \bibitem{programming}
about python\\
 \texttt{www.learnpython.org/}



\end{thebibliography}


\begin{center}
\newpage
\section{Epilogue}
\end{center}
\begin{enumerate}
\item{Inbuilt functions used for the string ,list ,dictionary behave differently for different python versions}
\item{Strings and tuples and immutable in python}
\end{enumerate}

\end{document} %end of document